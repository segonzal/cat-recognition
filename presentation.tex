\documentclass{beamer}
\usepackage[spanish, english]{babel}
\selectlanguage{spanish}
\usepackage[fixlanguage]{babelbib}
\selectbiblanguage{spanish}
\usepackage{graphicx}
\usepackage[utf8]{inputenc}%Para poder usar acentos y otras letras latinas del español
\usepackage[T1]{fontenc}

\usepackage{caption, subcaption}

\bibliographystyle{bababbrv}

\usetheme{Darmstadt}

\title{Reconocimiento facial de mascotas}
\subtitle{Enfocado en Perros y Gatos}
\author[Gonz\'alez]{S.~Gonz\'alez\inst{1}}
\institute[University de Chile]
{
	\inst{1}
	Departamento de las Ciencias de la Computaci\'on\\
	Universidad de Chile
}

\begin{document}
	\frame{\titlepage}
	\begin{frame}
		\frametitle{Motivaci\'on.}
		%CONTENT
		\begin{itemize}
			\item{
				Las mascotas son parte de la familia. Suelen extraviarse y encontrarlas es difícil.
			}
			\item{
				Aplicación en comederos automatizados, que reconoscan a la mascota.
			}
			\item{
				Vigilancia. Uso de perros de ataque/robo de animales.
			}
		\end{itemize}
	\end{frame}
	\begin{frame}
		\frametitle{Estado del Arte.}
		%CONTENT
		\selectlanguage{spanish}
		\selectbiblanguage{spanish}
		\bibliography{./informe}
	\end{frame}
	\begin{frame}
		\frametitle{Reconocimiento facial de gatos.}
		%CONTENT
		\url{http://harthur.github.io/kittydar/}

		Utiliza método propuesto en \emph{Cat head detection-how to effectively exploit shape and texture features}\cite{zhang2008cat}
		\begin{figure}[H]
			\centering
			\includegraphics[scale=0.27]{Captura.png}
		\end{figure}
	\end{frame}
	\begin{frame}
		\frametitle{Reconocimiento facial de gatos.}
		%CONTENT
		Comparación entre métodos.
		\includegraphics[scale=0.15]{annotation.png}
		\begin{itemize}
			\item{
				HOG.
			}
			\item{
				Haar.
			}
			\item{
				Detección de formas y textura conjunta.
			}
		\end{itemize}
		\begin{figure}[H]
			\centering
			\includegraphics[scale=0.30]{cat-roc.png}
		\end{figure}
	\end{frame}
	\begin{frame}
		\frametitle{Resultados}
		%CONTENT
		\begin{figure}[H]
			\centering
			\subcaptionbox{}{\includegraphics[scale=0.07]{nw1.jpg}}
			\subcaptionbox{}{\includegraphics[scale=0.38]{nw2.jpg}}
			\subcaptionbox{}{\includegraphics[scale=0.39]{nw3.jpg}}
			\caption{No reconoce gatitos.}
		\end{figure}
	\end{frame}
	\begin{frame}
		\frametitle{Resultados}
		%CONTENT
		\begin{figure}[H]
			\centering
			\subcaptionbox{}{\includegraphics[scale=0.3]{w1.png}}
			\subcaptionbox{}{\includegraphics[scale=0.31]{w2.png}}
			\caption{Reconoce algunos perros.}
		\end{figure}
	\end{frame}
	\begin{frame}
		\frametitle{Resultados}
		%CONTENT
		\begin{figure}[H]
			\centering
			\subcaptionbox{}{\includegraphics[scale=0.27]{f1.png}}
			\subcaptionbox{}{\includegraphics[scale=0.28]{f2.png}}
			\subcaptionbox{}{\includegraphics[scale=0.28]{f3.png}}
			\caption{Reconoce... cosas.}
		\end{figure}
	\end{frame}
	\begin{frame}
		\frametitle{Clasificación en Razas de Gatos y Perros.}
		%CONTENT
		\emph{Cats and Dogs}.\cite{parkhi12a}
		\begin{itemize}
			\item{
				HOG para reconocimiento de \emph{Forma}.
			}
			\item{
				\emph{Bag of Words} para reconocimiento de \emph{textura}.
			}
		\end{itemize}
		Combinaciones de Rostro, Cuerpo y Fondo.
		\begin{figure}[H]
			\centering
			\includegraphics[scale=0.50]{faceheadimage.png}
		\end{figure}
	\end{frame}
	\begin{frame}
		\frametitle{Clasificación en Razas de Gatos y Perros.}
		%CONTENT
		\begin{figure}[H]
			\centering
			{\includegraphics[scale=0.27]{datasetEx.png}}
			\caption{Dataset de razas de Perros y Gatos.}
		\end{figure}
	\end{frame}
	\begin{frame}
		\frametitle{Reconocimiento facial de Perros.}
		%CONTENT
		\emph{Biometric Recognition for Pet Animal}.\cite{kumar2014biometric}

		PCA \& Eigenfaces.
	\end{frame}
	\begin{frame}
		\frametitle{Reconocimiento facial de Perros.}
		%CONTENT
		\begin{figure}[H]
			\centering
			{\includegraphics[scale=0.27]{dogaverage.png}}
			\caption{Eigenfaces para Perros.}
		\end{figure}
	\end{frame}
\end{document}